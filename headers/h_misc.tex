% ------------------------------------------------------------------------------
% ---------------------------------  LICENSE  ----------------------------------
% ------------------------------------------------------------------------------
% Except where otherwise noted, this work by Noita Enola is licensed under CC BY-SA 4.0. You can check a copy of the license by looking at the LICENSE.md file, or at <https://creativecommons.org/licenses/by-sa/4.0>.

% ------------------------------------------------------------------------------
% These two speed up compilation time but produce a larger PDF. Should be commented before doing the final build: https://tex.stackexchange.com/a/454235
\pdfcompresslevel=0
\pdfobjcompresslevel=0

% ------------------------------------------------------------------------------
% Depth of the bookmarks
\setsecnumdepth{subsection}
\settocdepth{section}

% ------------------------------------------------------------------------------
% ToDo notes in the margins
\usepackage[textsize=scriptsize]{todonotes}
	\setlength{\marginparwidth}{2cm}

% ------------------------------------------------------------------------------
% Sections of text in multiple columns
\usepackage{multicol}
% ------------------------------------------------------------------------------
% To ease customizing of the list environments
\usepackage{enumitem}

% ------------------------------------------------------------------------------
\colorlet{colorLinks}{blue!30!black}

\usepackage[
	bookmarks,
	bookmarksopen,
	bookmarksdepth = 3,
	bookmarksopenlevel = 1,
	bookmarksnumbered,
%	hidelinks,
	colorlinks,
	allcolors=colorLinks,
	pdfusetitle,
	pdfview = Fit,
	pdfpagelayout = TwoPageRight,
]{hyperref}

% ------------------------------------------------------------------------------
\usepackage[newfloat]{minted}
\usepackage[skins, breakable, minted]{tcolorbox}
	\colorlet{bg}{black!5!white}
%	\definecolor{bg}{rgb}{0.95,0.95,0.95}
	\newcommand{\pythonScript}[1]{% Shortcut for inputing scripts
		\begin{tcolorbox}[%
			arc	= 0mm,
			colback	= bg,
			coltitle	= black,
			colframe	= bg,
			titlerule	= 1pt,
			titlerule style	= black,
			bottomrule	= 1pt,
			title	= #1,
			fonttitle	=\sffamily\bfseries,
			breakable,
			title after break	= #1 (cont.),
			toprule at break	= 0pt,
			bottomrule at break = 0pt,
			]%
			\inputminted[breaklines, breakafter=/, fontfamily=tt]{python}{./scripts/#1}%
		\end{tcolorbox}%
	}

% ------------------------------------------------------------------------------
% To include the CC logo
\usepackage[
	type		= {CC},
	modifier	= {by-sa},
	version		= {4.0},
]{doclicense}

% ------------------------------------------------------------------------------
% Bibliography handling.
\usepackage[style=numeric,sorting=none,sortcites]{biblatex}
\addbibresource{bibliography.bib}
%\renewcommand{\cite}[1]{\textbf{[#1]}}

% ------------------------------------------------------------------------------
% To generate the nicer folder trees. By Guilherme Zanotelli: https://tex.stackexchange.com/questions/5073/making-a-simple-directory-tree
\usepackage[edges]{forest}
\definecolor{foldercolor}{RGB}{124,166,198}
\tikzset{pics/folder/.style={code={%
			\node[inner sep=0pt, minimum size=#1](-foldericon){};
			\node[folder style, inner sep=0pt, minimum width=0.3*#1, minimum height=0.6*#1, above right, xshift=0.05*#1] at (-foldericon.west){};
			\node[folder style, inner sep=0pt, minimum size=#1] at (-foldericon.center){};}
	},
	pics/folder/.default={20pt},
	folder style/.style={draw=foldercolor!80!black,top color=foldercolor!40,bottom color=foldercolor}
}

\forestset{is file/.style={edge path'/.expanded={%
			([xshift=\forestregister{folder indent}]!u.parent anchor) |- (.child anchor)},
		inner sep=1pt},
	this folder size/.style={edge path'/.expanded={%
			([xshift=\forestregister{folder indent}]!u.parent anchor) |- (.child anchor) pic[solid]{folder=#1}}, inner xsep=0.6*#1},
	folder tree indent/.style={before computing xy={l=#1}},
	folder icons/.style={folder, this folder size=#1, folder tree indent=3*#1},
	folder icons/.default={12pt},
}

% ------------------------------------------------------------------------------
% Package for the menus and buttons
\usepackage[os=win]{menukeys}
\renewmenumacro{\directory}[/]{paths}
\renewmenumacro{\keys}[+]{shadowedroundedkeys}

% ------------------------------------------------------------------------------
% Add some space between number and text in the footnotes. By JohnReed: https://tex.stackexchange.com/questions/54685/inserting-space-after-the-number-in-footnotes
\let\oldfootnote\footnote
\renewcommand\footnote[1]{\oldfootnote{\hspace{0.5em}#1}}

\let\oldfootnotetext\footnotetext
\renewcommand\footnotetext[1]{\oldfootnotetext{\hspace{0.5em}#1}}
% ------------------------------------------------------------------------------
