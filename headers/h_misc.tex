% ------------------------------------------------------------------------------
% ---------------------------------  LICENSE  ----------------------------------
% ------------------------------------------------------------------------------
% Except where otherwise noted, this work by Noita Enola is licensed under CC BY-SA 4.0. You can check a copy of the license by looking at the LICENSE.md file, or at <https://creativecommons.org/licenses/by-sa/4.0>.

% ------------------------------------------------------------------------------
% These two speed up compilation time but produce a larger PDF. Should be commented before doing the final build: https://tex.stackexchange.com/a/454235
\pdfcompresslevel=0
\pdfobjcompresslevel=0

% ------------------------------------------------------------------------------
% ToDo notes in the margins
%\usepackage[textsize=scriptsize]{todonotes}
%	\setlength{\marginparwidth}{2cm}

% ------------------------------------------------------------------------------
% Sections of text in multiple columns
\usepackage{multicol}

% ------------------------------------------------------------------------------
% To ease customizing of list environments
\usepackage{enumitem}

% ------------------------------------------------------------------------------
% Nice code snippets typestting
\usepackage[skins, breakable, minted]{tcolorbox}
	\colorlet{bg}{black!5!white} % background color for snippets
%	\definecolor{bg}{rgb}{0.95,0.95,0.95}
	\newcommand{\pythonScript}[3]{% Shortcut for inputting python scripts
		\begin{tcolorbox}[%
			arc	= 0mm,
			colback				= bg,
			coltitle			= black,
			colframe			= bg,
			titlerule			= 1pt,
			titlerule style		= black,
			bottomrule			= 1pt,
			title				= #1,
			fonttitle			=\sffamily\bfseries,
			breakable			= true,
			title after break	= #1 (cont.),
			toprule at break	= 0pt,
			bottomrule at break	= 0pt,
%			code				= {},
			]%
			\captionsetup{type=script}% Putting \captionsetup outside the tcolorbox (or using code = {...}) misplaces some anchors in the previous pages when the snippet is the first thing of the page. Putting it here makes the links point just below the script title.
			\inputminted[
				breaklines,
				breakafter	= /,
				fontfamily	= tt,
				tabsize	= 4,
			]{python}{./scripts/#1}%
			\vspace{-0.5\baselineskip}
			\caption{#2}
		\end{tcolorbox}%
	}

% ------------------------------------------------------------------------------
% Support for putting documents under a license
\usepackage[
	type		= {CC},
	modifier	= {by-sa},
	version		= {4.0},
]{doclicense}

% ------------------------------------------------------------------------------
% Advanced bibliographic facilities
\usepackage[style=numeric,sorting=none,sortcites]{biblatex}
	\addbibresource{bibliography.bib}
	%\renewcommand{\cite}[1]{\textbf{[#1]}}

% ------------------------------------------------------------------------------
% Nice, simple directory trees. By Guilherme Zanotelli: https://tex.stackexchange.com/questions/5073/making-a-simple-directory-tree
\usepackage[edges]{forest}
\definecolor{foldercolor}{RGB}{124,166,198}
\tikzset{%
	pics/folder/.style = {%
		code = {%
			\node[inner sep=0pt, minimum size=#1](-foldericon){};
			\node[folder style, inner sep=0pt, minimum width=0.3*#1, minimum height=0.6*#1, above right, xshift=0.05*#1] at (-foldericon.west){};
			\node[folder style, inner sep=0pt, minimum size=#1] at (-foldericon.center){};
		}
	},
	pics/folder/.default = {20pt},
	folder style/.style = {%
		draw = foldercolor!80!black,
		top color = foldercolor!40,
		bottom color = foldercolor
	}
}

\forestset{%
	is file/.style = {%
		edge path'/.expanded = {%
			([xshift = \forestregister{folder indent}]!u.parent anchor) |- (.child anchor)
		},
		inner sep = 1pt
	},
	this folder size/.style = {%
		edge path'/.expanded = {%
			([xshift=\forestregister{folder indent}]!u.parent anchor) |- (.child anchor) pic[solid]{folder=#1}
		},
		inner xsep = 0.6*#1
	},
	folder tree indent/.style = {%
		before computing xy = {l=#1}
	},
	folder icons/.style = {%
		folder,
		this folder size = #1,
		folder tree indent = 3*#1
	},
	folder icons/.default = {12pt},
}

% ------------------------------------------------------------------------------
% Menu sequences, paths and keystrokes formatting
\usepackage[os=win]{menukeys}
	\renewmenumacro{\directory}[/]{paths}
	\renewmenumacro{\keys}[+]{shadowedroundedkeys}

% ------------------------------------------------------------------------------
% Hypertext support for LaTeX

\usepackage[
	bookmarks			= true,
	bookmarksopen		= true,
	bookmarksdepth		= 3,
	bookmarksopenlevel	= 0,
	bookmarksnumbered	= true,
%	hidelinks			= true,
	colorlinks			= true,
	allcolors			= blue!30!black,
	pdfusetitle			= true,
	pdfstartview		= Fit,
	pdfpagelayout		= TwoPageRight,
]{hyperref}
